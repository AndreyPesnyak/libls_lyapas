\documentclass[12t]{article}

\usepackage [utf8x] {inputenc}
\usepackage [T2A] {fontenc}
\usepackage {xcolor}
\usepackage {hyperref}
\usepackage {float}
\usepackage {longtable}
\usepackage [english,russianb]{babel}
\definecolor{color1}{HTML}{007FFF}
\hypersetup{pdfstartview=FitH, linkcolor=color1, colorlinks=true}
\begin{document}

\textcolor[rgb]{1,0,0}{server.l}

Os Oe Oc Od Ot Op ; обнуление переменных

@+F1(16) @+F2(16) @+F3(16) ; 1 комлпекс - для настроек сервера, 2 - для настроек клиента, 3 - для обмена сообщением

@'0.0.0.0'>F2

2⇒d 1⇒t 0⇒p ; domain=AF\_INET type=SOCK\_STREAM protocol=auto

\hyperref [set_sockaddr_bind_domain] {*set\_sockaddr\_bind\_domain(F1,2/)} ;настройка сервера domain=AF\_INET 

\hyperref [set_sockaddr_bind_port] {*set\_sockaddr\_bind\_port(F1,1195/)} ;настройка сервера port=1195

\hyperref [set_sockaddr_bind_ip] {*set\_sockaddr\_bind\_ip(F1,F2/)} ;настройка сервера IP=любой

\hyperref [socket] {*socket(d,t,p/s)}

\hyperref [bind] {*bind(s,F1,10h/e)} ;привязываем сокет s к комплексу F1 длины 16 байт

\hyperref [listen] {*listen(s,5/e)} ;ожидаем не более 5 соединений на сокете s

\hyperref [accept] {*accept(s,F2,10h/c)} ;устанавливаем связь на сокете s

*freadf(c,F3,0,5/a) ;считываем 5 символов с клиентского сокета в комплекс F3

*fwritef(c,F3,0,5/a) ;записываем 5 символов в клиентский сокет из комплекса F3

*close(c/) ;закрываем сокет клиента

*close(s/) ;закрываем сокет сервера

**
~\\
~\\
\label{socket}
socket(\hyperref[domain]{domain}, \hyperref[type]{type}, \hyperref[protocol]{protocol}/socket) - эта функция используется для создания сокета. Первый параметр - домен - накладывает определенные ограничения на формат используемых процессов адресов и их интерпретацию. Второй параметр определяет тип канала связи с сокетом, который должен быть использован. Третий параметр позволяет выбрать нужный протокол для канала связи; если он равен 0, то операционная система выбирает его автоматически. Функция возвращает целое положительное число - номер дискриптора или -1 в случае ошибки.

\begin{table}[H]
\caption{\label{domain}Значения domain}
\begin{center}
\begin{tabular}{|c|c|}
\hline
domain & значение \\
\hline
AF\_UNIX, AF\_LOCAL & 1 \\
AF\_INET & 2 \\
AF\_INET6 & 10 \\
AF\_IPX & 4 \\
AF\_NETLINK & 16 \\
AF\_X25 & 9 \\
AF\_AX25 & 3 \\
AF\_ATMPVC & 8 \\
AF\_APPLETALK & 5 \\
AF\_PACKET & 17 \\
\hline
\end{tabular}
\end{center}
\end{table}

\begin{table}[ht]
\caption{\label{type}Значения type}
\begin{center}
\begin{tabular}{|c|c|}
\hline
type & значение \\
\hline
SOCK\_STREAM & 1 \\
SOCK\_DGRAM & 2 \\
SOCK\_SEQPACKET & 5 \\
SOCK\_RAW & 3 \\
SOCK\_RDM & 4 \\
SOCK\_PACKET & 10 \\
SOCK\_NONBLOCK & 2048 \\
SOCK\_CLOEXEC & 524288 \\
\hline
\end{tabular}
\end{center}
\end{table}

\begin{table}[H]
\caption{\label{protocol}Значения protocol}
\begin{center}
\begin{tabular}{|c|c|}
\hline
protocol & значения \\
\hline
IPPROTO\_IP & 0 \\
IPPROTO\_HOPOPTS & 0 \\
IPPROTO\_ICMP & 1 \\
IPPROTO\_IGMP & 2 \\
IPPROTO\_IPIP & 4 \\
IPPROTO\_TCP & 6 \\
IPPROTO\_EGP & 8 \\
IPPROTO\_PUP & 12 \\
IPPROTO\_UDP & 17 \\
IPPROTO\_IDP & 22 \\
IPPROTO\_TP & 29 \\
IPPROTO\_DCCP & 33 \\
IPPROTO\_IPV6 & 41 \\
IPPROTO\_ROUTING & 43 \\
IPPROTO\_FRAGMENT & 44 \\
IPPROTO\_RSVP & 46 \\
IPPROTO\_GRE & 47 \\
IPPROTO\_ESP & 50 \\
IPPROTO\_AH & 51 \\
IPPROTO\_ICMPV6 & 58 \\
IPPROTO\_NONE & 59 \\
IPPROTO\_DSTOPTS & 60 \\
IPPROTO\_MTP & 92 \\
IPPROTO\_ENCAP & 98 \\
IPPROTO\_PIM & 103 \\
IPPROTO\_COMP & 108 \\
IPPROTO\_SCTP & 132 \\
IPPROTO\_UDPLITE & 136 \\
IPPROTO\_RAW & 255 \\
\hline
\end{tabular}
\end{center}
\end{table}

\phantomsection
\label{bind}
bind(socket,\hyperref[sockaddr]{sockaddr}, sockaddr\_len/\hyperref[error]{error}) - используется сервером для присваивания сокету имени. До выполнения этой функции сокет недоступен программам-клиентам. Первый аргумент - сокет-дескриптор, который данная функция именует. Второй параметр - указатель на комплекс, описывающий sockaddr. Третий параметр - длина комплекса. В случае успешного выполнения функция возвращает значение равное 0, иначе - код ошибки.

\phantomsection
\label{set_sockaddr_bind_domain}
set\_sockaadr\_bind\_domain(\hyperref[sockaddr]{sockaddr},\hyperref[domain]{domain}) - функция, позволяющая заполнить поле комплекса sockaddr значением domain.

\phantomsection
\label{set_sockaddr_bind_port}
set\_sockaddr\_bind\_port(\hyperref[sockaddr]{sockaddr},port) - функция, позволяющая заполнить поле комплекса sockaddr значением port.

\phantomsection
\label{set_sockaddr_bind_ip}
set\_sockaddr\_bind\_ip(\hyperref[sockaddr]{sockaddr},ip) - функция, позволяющая заполнить поле комплекса sockaddr значением, взятым из комплекса ip. В случае, если комплекс заполнен нулями, для установления соединения будут использованы все IP-адреса сервера.

\begin{table}[H]
\caption{\label{sockaddr}Комплекс sockaddr}
\begin{center}
\begin{tabular}{|c|c|c|c|c|c|c|c|c|c|c|c|c|c|c|c|c|}
\hline
биты & 0 & 1 & 2 & 3 & 4 & 5 & 6 & 7 & 8 & 9 & 10 & 11 & 12 & 13 & 14 & 15 \\
\hline
значение & \multicolumn{2}{|c|}{domain} & \multicolumn{2}{|c|}{port} & \multicolumn{4}{|c|}{IP address} & 0 & 0 & 0 & 0 & 0 & 0 & 0 & 0 \\
\hline
\end{tabular}
\end{center}
\end{table}

\phantomsection
\label{listen}
listen(socket, backlog/\hyperref[error]{error}) - используется сервером, чтобы уведомить операционную систему, что он ожидает запросы на установление соединения на данном сокете. Первый аргумент - сокет для прослушивания, второй аргумент - целое положительное число, определяющее как много запросов может быть принято на сокет. В случае успешного выполнения функция возвращает значение равное 0, иначе - код ошибки.

\phantomsection
\label{accept}
accept(socket,\hyperref[sockaddr]{sockaddr},sockaddr\_len/socket) - используется сервером для принятия соединения на сокет. Сокет должен быть уже слушающим в момент вызова функции. Если сервер устанавливает соединение с клиентом, то функция accept возвращает новый сокет-дескриптор, через который и происходит общение клиента с сервером, в противном случае -1. Пока устанавливается соединение клиента с сервером, функция accept блокирует другие запросы соединения с данным сервером, а после установления связи "прослушивание" запросов возобновляется. Первый аргумент функции - сокет-дескриптор для принятия соединений от клиентов. Второй аргумент - комплекс sockaddr для соответствующего домена. Третий аргумент - длина комплекса sockaddr. Второй и третий аргументы заполняются соответствующими значениями в момент установления соединения клиента с сервером и позволяют серверу точно определить, с каким именно клиентом он общается.

\phantomsection
\label{connect}
connect(socket,\hyperref[sockaddr]{sockaddr},sockaddr\_len/\hyperref[error]{error}) - данная функция используется клиентом для установления соединения с сервером. Первый аргумент - сокет клиента, второй - комплекс sockaddr, соответствующий серверу, третий - длина комплекса sockaddr. Если соединение прошло успешно возвращается значение равное нулю, иначе - код ошибки.

\begin{center}
\begin{longtable}{|c|c|}
\caption{\label{error}Значения error}
\\ \hline
Значение & Обозначение ошибки \\
\hline
1 & операция не разрешена \\
2 & файл или директория не существует \\
3 & процесс не существует \\
4 & прерывание системы \\
5 & ошибка ввода/вывода \\
6 & устройство или адрес не существует \\
7 & аргументов слишком много \\
8 & ошибка формата исполняемого файла \\
9 & некорректный номер дескриптора \\
10 & дочернего процесса не существует \\
11 & перезапустите процесс \\
12 & ошибка памяти \\
13 & доступ запрещен \\
14 & некорректный адрес \\
15 & требуется блочное устройство \\
16 & устройство или ресурс занято\\
17 & файл существует\\
18 & нельзя создавать жесткие ссылки между устройствами\\
19 & устройство не существует\\
20 & не является директорией\\ 
21 & является директорией\\
22 & неверный аргумент \\
23 & переполнение таблицы дескрипторов \\
24 & много открытых файлов \\
25 & не является пишущим устройством\\
26 & текстовый файл занят\\
27 & файл слишком большой \\
28 & на устройстве не хватает свободного места \\
29 & запрещено передвижение указателя на файл\\
30 & доступны только права на чтение\\
31 & слишком много ссылок\\
32 & ошибка ввода данных\\
33 & результат математической операции нельзя представить\\
\hline
\end{longtable}
\end{center}
\end{document}
