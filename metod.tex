\documentclass[12t,english,russian]{article}

\usepackage [utf8x] {inputenc}
\usepackage [T2A] {fontenc}
\usepackage {geometry}
\usepackage {xcolor}
\usepackage {hyperref}
\usepackage {multirow}
\usepackage {float}
\usepackage {longtable}
\usepackage [english,russianb]{babel}
\geometry {top=2cm,bottom=2cm,left=2cm,right=2cm}
\definecolor{color1}{HTML}{007FFF}
\hypersetup{pdfstartview=FitH, linkcolor=color1, colorlinks=true}
\parindent=1cm

\begin{document}
\huge
\begin{center}
{Сетевое взаимодействие на языке Ляпас}
\end{center}
\Large
\tableofcontents
\newpage

\large
\section[Описание функций]{Описание функций}
\phantomsection
\label{socket}
\hspace{\parindent}
socket(\hyperref[domain]{domain}, \hyperref[type]{type}, \hyperref[protocol]{protocol}/socket) - эта функция используется для создания сокета. Первый параметр - домен - накладывает определенные ограничения на формат используемых процессов адресов и их интерпретацию. Второй параметр определяет тип канала связи с сокетом, который должен быть использован. Третий параметр позволяет выбрать нужный протокол для канала связи; если он равен 0, то операционная система выбирает его автоматически. Функция возвращает целое положительное число - номер дискриптора или -1 в случае ошибки.

\begin{table}[H]
\caption{\label{domain}Значения domain}
\begin{center}
\begin{tabular}{|c|c|}
\hline
domain & значение \\
\hline
AF\_UNIX, AF\_LOCAL & 1 \\
AF\_INET & 2 \\
AF\_INET6 & 10 \\
AF\_IPX & 4 \\
AF\_NETLINK & 16 \\
AF\_X25 & 9 \\
AF\_AX25 & 3 \\
AF\_ATMPVC & 8 \\
AF\_APPLETALK & 5 \\
AF\_PACKET & 17 \\
\hline
\end{tabular}
\end{center}
\end{table}

\begin{table}[ht]
\caption{\label{type}Значения type}
\begin{center}
\begin{tabular}{|c|c|}
\hline
type & значение \\
\hline
SOCK\_STREAM & 1 \\
SOCK\_DGRAM & 2 \\
SOCK\_SEQPACKET & 5 \\
SOCK\_RAW & 3 \\
SOCK\_RDM & 4 \\
SOCK\_PACKET & 10 \\
SOCK\_NONBLOCK & 2048 \\
SOCK\_CLOEXEC & 524288 \\
\hline
\end{tabular}
\end{center}
\end{table}

\begin{table}[H]
\caption{\label{protocol}Значения protocol}
\begin{center}
\begin{tabular}{|c|c|}
\hline
protocol & значения \\
\hline
IPPROTO\_IP & 0 \\
IPPROTO\_HOPOPTS & 0 \\
IPPROTO\_ICMP & 1 \\
IPPROTO\_IGMP & 2 \\
IPPROTO\_IPIP & 4 \\
IPPROTO\_TCP & 6 \\
IPPROTO\_EGP & 8 \\
IPPROTO\_PUP & 12 \\
IPPROTO\_UDP & 17 \\
IPPROTO\_IDP & 22 \\
IPPROTO\_TP & 29 \\
IPPROTO\_DCCP & 33 \\
IPPROTO\_IPV6 & 41 \\
IPPROTO\_ROUTING & 43 \\
IPPROTO\_FRAGMENT & 44 \\
IPPROTO\_RSVP & 46 \\
IPPROTO\_GRE & 47 \\
IPPROTO\_ESP & 50 \\
IPPROTO\_AH & 51 \\
IPPROTO\_ICMPV6 & 58 \\
IPPROTO\_NONE & 59 \\
IPPROTO\_DSTOPTS & 60 \\
IPPROTO\_MTP & 92 \\
IPPROTO\_ENCAP & 98 \\
IPPROTO\_PIM & 103 \\
IPPROTO\_COMP & 108 \\
IPPROTO\_SCTP & 132 \\
IPPROTO\_UDPLITE & 136 \\
IPPROTO\_RAW & 255 \\
\hline
\end{tabular}
\end{center}
\end{table}

\phantomsection
\label{bind}
bind(\hyperref[socket]{socket},\hyperref[sockaddr]{sockaddr}, sockaddr\_len/\hyperref[error]{error}) - используется сервером для присваивания сокету имени. До выполнения этой функции сокет недоступен программам-клиентам. Первый аргумент - сокет-дескриптор, который данная функция именует. Второй параметр - указатель на комплекс, описывающий sockaddr. Третий параметр - длина комплекса. В случае успешного выполнения функция возвращает значение равное 0, иначе - код ошибки.

\begin{table}[H]
\caption{\label{sockaddr}Комплекс sockaddr}
\begin{center}
\begin{tabular}{|c|c|c|c|c|c|c|c|c|c|c|c|c|c|c|c|c|}
\hline
биты & 0 & 1 & 2 & 3 & 4 & 5 & 6 & 7 & 8 & 9 & 10 & 11 & 12 & 13 & 14 & 15 \\
\hline
значение & \multicolumn{2}{|c|}{domain} & \multicolumn{2}{|c|}{port} & \multicolumn{4}{|c|}{IP address} & 0 & 0 & 0 & 0 & 0 & 0 & 0 & 0 \\
\hline
\end{tabular}
\end{center}
\end{table}

\phantomsection
\label{set_sockaddr_domain}
set\_sockaadr\_domain(\hyperref[sockaddr]{sockaddr},\hyperref[domain]{domain}) - функция, позволяющая заполнить поле комплекса sockaddr значением domain.

\phantomsection
\label{set_sockaddr_port}
set\_sockaddr\_port(\hyperref[sockaddr]{sockaddr},port) - функция, позволяющая заполнить поле комплекса sockaddr значением port.

\phantomsection
\label{set_sockaddr_ip}
set\_sockaddr\_ip(\hyperref[sockaddr]{sockaddr},ip) - функция, позволяющая заполнить поле комплекса sockaddr значением, взятым из комплекса ip. В случае, если комплекс заполнен нулями, для установления соединения будут использованы все IP-адреса сервера.

\phantomsection
\label{listen}
listen(\hyperref[socket]{socket}, backlog/\hyperref[error]{error}) - используется сервером, чтобы уведомить операционную систему, что он ожидает запросы на установление соединения на данном сокете. Первый аргумент - сокет для прослушивания, второй аргумент - целое положительное число, определяющее как много запросов может быть принято на сокет. В случае успешного выполнения функция возвращает значение равное 0, иначе - код ошибки.

\phantomsection
\label{accept}
accept(\hyperref[socket]{socket},\hyperref[sockaddr]{sockaddr},sockaddr\_len/\hyperref[socket]{socket}) - используется сервером для принятия соединения на сокет. Сокет должен быть уже слушающим в момент вызова функции. Если сервер устанавливает соединение с клиентом, то функция accept возвращает новый сокет-дескриптор, через который и происходит общение клиента с сервером, в противном случае -1. Пока устанавливается соединение клиента с сервером, функция accept блокирует другие запросы соединения с данным сервером, а после установления связи "прослушивание" запросов возобновляется. Первый аргумент функции - сокет-дескриптор для принятия соединений от клиентов. Второй аргумент - комплекс sockaddr для соответствующего домена. Третий аргумент - длина комплекса sockaddr. Второй и третий аргументы заполняются соответствующими значениями в момент установления соединения клиента с сервером и позволяют серверу точно определить, с каким именно клиентом он общается.

\phantomsection
\label{connect}
connect(\hyperref[socket]{socket},\hyperref[sockaddr]{sockaddr},sockaddr\_len/\hyperref[error]{error}) - данная функция используется клиентом для установления соединения с сервером. Первый аргумент - сокет клиента, второй - комплекс sockaddr, соответствующий серверу, третий - длина комплекса sockaddr. Если соединение прошло успешно возвращается значение равное нулю, иначе - код ошибки.

\begin{center}
\begin{longtable}{|c|c|}
\caption{\label{error}Значения error}
\\ \hline
Значение & Описание ошибки \\
\hline
1 & Функция заблокирована межсетевым экраном\\
2 & Сокета с заданным номером не существует\\
4 & Системный вызов был прерван сигналом\\
9 & Некорректный номер дескриптора\\
11 & В данный момент функция не может быть выполнена, попробуйте позже\\
12 & Недостаточно свободного места для создания сокета \\
13 & Запрещено создание сокета с заданными параметрами \\
14 & Некорректный логический адрес параметра\\
20 & Аргумент не является директорией\\
22 & Некорректное значение аргумента\\
23 & Переполнение таблицы дескрипторов\\
24 & Слишком много открытых файлов\\
30 & Файловая система доступна только на чтение\\
36 & Комплекс sockaddr слишком длинный\\
40 & Слишком много символических ссылок \\
71 & Ошибка протокола\\
88 & Значение аргумента socket, не является дескриптором\\
91 & Тип сокета не поддерживает протокол,выбранный сервером\\
93 & Данный тип протокола не поддерживается\\
95 & Функция не поддерживает данные типы сокетов\\
97 & Семейство адресов не поддерживается протоколом\\
98 & Сетевой адрес уже используется\\
99 & Требуемый сетевой адрес не может быть использован\\
101 & Сеть недоступна\\
\hline
103 & Соединение разорвано\\
105 & Недостаточно свободного места для создания сокета\\
106 & Соединение на сокете уже произошло\\
110 & Истечение времени ожидания\\
111 & Не обнаружен слушающий порт сервера\\
\multirow{2}{*}{114} & Сокет является неблокирующим, а предыдущая\\
& попытка установить соединение ещё не завершена\\
\multirow{2}{*}{115} & Сокет является неблокирующим, а соединение\\
& не может быть установлено в данный момент\\
\hline
\end{longtable}
\end{center}
\section[Примеры использования]{Примеры использования}
\begin{center}
\textcolor[rgb]{1,0,0}{Эхо-сервер}
\end{center}

main(/)

\ \ \ \ Os Oe Oq Oc Oy Or Od Ot Op

\ \ \ \ @+F1(16) @+F2(16) @+F3(16) @+F4(10)

\ \ \ \ @'0.0.0.0'>F2

\ \ \ \ 2⇒d 1⇒t 0⇒p

\ \ \ \ \hyperref[set_sockaddr_domain]{*set\_sockaddr\_domain}(F1,2/)

\ \ \ \ \hyperref[set_sockaddr_port]{*set\_sockaddr\_port}(F1,1195/)

\ \ \ \ \hyperref[set_sockaddr_ip]{*set\_sockaddr\_ip}(F1,F2/)

\ \ \ \ \hyperref[socket]{*socket}(d,t,p/s)

\ \ \ \ \hyperref[bind]{*bind}(s,F1,10h/e)

\ \ \ \ \hyperref[listen]{*listen}(s,100/e)

§1 \hyperref[accept]{*accept}(s,F3,10h/c)

\ \ \ \ *freadf(c,F4,0,5/a)	

\ \ \ \ *fwritef(c,F4,0,5/a)

\ \ \ \ *close(c/)

\ \ \ \ →1

\ \ \ \ *close(s/)

\ \ \ \ **

\begin{center}
\textcolor[rgb]{1,0,0}{Эхо-клиент}
\end{center}
main(/)

\ \ \ \ Ox Oq Oy Oe Od Ot Op

\ \ \ \ @+F1(16) @+F2(16) @+F3(5) @+F4(5)

\ \ \ \ @'192.168.203.1'>F2

\ \ \ \ @'echo\n'>F3

\ \ \ \ 2⇒d 1⇒t 0⇒p

\ \ \ \ \hyperref[set_sockaddr_domain]{*set\_sockaddr\_domain}(F1,2/)

\ \ \ \ \hyperref[set_sockaddr_port]{*set\_sockaddr\_port}(F1,1195/)

\ \ \ \ \hyperref[set_sockaddr_ip]{*set\_sockaddr\_ip}(F1,F2/)

\ \ \ \ \hyperref[socket]{*socket}(d,t,p/x)

\ \ \ \ \hyperref[connect]{*connect}(x,F1,10h/e)

\ \ \ \ *fwritef(x,F3,0,5/a)

\ \ \ \ *freadf(x,F4,0,5/a)

\ \ \ \ *fwritef(1,F4,0,5/a)

\ \ \ \ *close(x/)

\ \ \ \ **
\end{document}
