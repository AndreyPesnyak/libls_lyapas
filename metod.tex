\documentclass[12t,english,russian]{article}

\usepackage [utf8x] {inputenc}
\usepackage [T2A] {fontenc}
\usepackage {geometry}
\usepackage {xcolor}
\usepackage {hyperref}
\usepackage {multirow}
\usepackage {float}
\usepackage {longtable}
\usepackage [english,russianb]{babel}
\geometry {top=2cm,bottom=2cm,left=2cm,right=2cm}
\definecolor{color1}{HTML}{007FFF}
\hypersetup{pdfstartview=FitH, linkcolor=color1, colorlinks=true}
\parindent=1cm

\begin{document}
\huge
\begin{center}
{Сетевое взаимодействие на языке Ляпас}
\end{center}
\Large
\tableofcontents
\newpage

\large
\section[Описание функций]{Описание функций}
\phantomsection
\label{socket}
\hspace{\parindent}
socket(\hyperref[type]{t}/s) - создания сокета. Исходные данные: t - тип логического соединения. Результат - номер дексприптора или $2^{32}-1$ в случае отказа

\phantomsection
\label{bind}
bind(\hyperref[socket]{s},\hyperref[sockaddr]{F1}/\hyperref[error]{e}) - назначение сокету локального адреса. Исходные данные: s - дескриптор сокета, F1 - комплекс, задающий структуру сокетного адреса. Результат - в случае успешного выполнения подпрограмма возвращает значение равное 0, иначе - код ошибки.

\phantomsection
\label{set_sockaddr_port}
set\_sockaddr\_port(\hyperref[sockaddr]{F1},p/) - установление значения порта в сокетном адресе. Исходные данные: F1 - комплекс длиной 16 символов, задающий структуру сокетного адреса, p - номер порта.

\phantomsection
\label{set_sockaddr_ip}
set\_sockaddr\_ip(\hyperref[sockaddr]{F1},F2/) - установление значения сетевого адреса в сокетном адресе. Исходные данные: F1 - комплекс длиной 16 символов, задающий структуру сокетного адреса, F2 - комплекс, задающий сетевой адрес, записанный в десятичном виде с разделительным символом "." между байтами.

\phantomsection
\label{listen}
listen(\hyperref[socket]{s},b/\hyperref[error]{e}) - установка сокета в режим "прослушивания". Исходные данные: s - дескриптор сокета для прослушивания, b - целое положительное число, определяющее как много запросов может быть принято на сокет. Результат - в случае успешного выполнения подпрограмма возвращает значение равное 0, иначе - код ошибки.

\phantomsection
\label{accept}
accept(\hyperref[socket]{s},\hyperref[sockaddr]{F1}/\hyperref[socket]{s}) - принятиe соединения на сокет. Исходные данные: s - дескриптор слушающего сокета, F1 - комплекс, описывающий сокетный адрес клиента. Результат - номер дескриптора или $2^{32}-1$ в случае отказа.

get\_sockaddr\_ip(\hyperref[sockaddr]{F1}/F2) - взятие значения сетевого адреса из сокетного адреса. Исходные данные: F1 - комплекс, задающий сокетный адрес, F2 - комплекс, задающий сетевой адрес. Результат - в комплексе F2 будет записан сетевой адрес с использованием разделительного символа "." между байтами сетевого адреса.

get\_sockaddr\_port(\hyperref[sockaddr]{F1}/p) - взятие значения порта из сокетного адреса. Исходные данные: F1 - комплекс, задающий сокетный адрес. Результат - номер порта.

\phantomsection
\label{connect}
connect(\hyperref[socket]{s},\hyperref[sockaddr]{F1}/\hyperref[error]{e}) - установление соединения с сервером. Исходные данные: s - дескриптор сокета, F1 - комплекс, задающий сокетный адрес сервера. Результат - значение 0 или код ошибки.

\begin{table}[H]
\caption{\label{sockaddr}Структура сокетного адреса}
\begin{center}
\begin{tabular}{|c|c|c|c|c|c|c|c|c|c|c|c|c|c|c|c|c|}
\hline
номер байта & 0 & 1 & 2 & 3 & 4 & 5 & 6 & 7 & 8 & 9 & 10 & 11 & 12 & 13 & 14 & 15 \\
\hline
значение & 2 & 0 & \multicolumn{2}{|c|}{port} & \multicolumn{4}{|c|}{IP address} & 0 & 0 & 0 & 0 & 0 & 0 & 0 & 0 \\
\hline
\end{tabular}
\end{center}
\end{table}

\begin{table}[ht]
\caption{\label{type}Значения t}
\begin{center}
\begin{tabular}{|c|c|}
\hline
t & значение \\
\hline
с установление логического соединения & 1 \\
без установления логического соединения & 2 \\
\hline
\end{tabular}
\end{center}
\end{table}

\newpage
\begin{center}
\begin{longtable}{|c|c|}
\caption{\label{error}Значения e}
\\ \hline
Значение & Описание ошибки \\
\hline
1 & Функция заблокирована межсетевым экраном\\
2 & Сокета с заданным номером не существует\\
4 & Системный вызов был прерван сигналом\\
9 & Некорректный номер дескриптора\\
11 & В данный момент функция не может быть выполнена, попробуйте позже\\
12 & Недостаточно свободного места для создания сокета \\
13 & Запрещено создание сокета с заданными параметрами \\
14 & Некорректный логический адрес параметра\\
22 & Некорректное значение аргумента\\
23 & Переполнение таблицы дескрипторов\\
24 & Слишком много открытых файлов\\
30 & Файловая система доступна только на чтение\\
36 & Комплекс F1 слишком длинный\\
71 & Ошибка протокола\\
88 & Значение аргумента s, не является дескриптором\\
91 & Тип сокета не поддерживает протокол,выбранный сервером\\
93 & Данный тип протокола не поддерживается\\
95 & Функция не поддерживает данные типы сокетов\\
98 & Сетевой адрес уже используется\\
99 & Требуемый сетевой адрес не может быть использован\\
101 & Сеть недоступна\\
103 & Соединение разорвано\\
105 & Недостаточно свободного места для создания сокета\\
106 & Соединение на сокете уже произошло\\
110 & Истечение времени ожидания\\
111 & Не обнаружен слушающий порт сервера\\
\multirow{2}{*}{114} & Сокет является неблокирующим, а предыдущая\\
& попытка установить соединение ещё не завершена\\
\multirow{2}{*}{115} & Сокет является неблокирующим, а соединение\\
& не может быть установлено в данный момент\\
\hline
\end{longtable}
\end{center}
\newpage
\section[Примеры использования]{Примеры использования}
\begin{center}
\textcolor[rgb]{1,0,0}{Эхо-сервер}
\end{center}

main(/)

\ \ \ \ Os Oe Oc Oa

\ \ \ \ @+F1(16) @+F2(16) @+F3(16) @+F4(10)

\ \ \ \ @'0.0.0.0'>F2

\ \ \ \ \hyperref[set_sockaddr_port]{*set\_sockaddr\_port}(F1,1195/)

\ \ \ \ \hyperref[set_sockaddr_ip]{*set\_sockaddr\_ip}(F1,F2/)

\ \ \ \ \hyperref[socket]{*socket}(1/s)

\ \ \ \ \hyperref[bind]{*bind}(s,F1/e)

\ \ \ \ \hyperref[listen]{*listen}(s,100/e)

§1 \hyperref[accept]{*accept}(s,F3/c)

\ \ \ \ *freadf(c,F4,0,5/a)	

\ \ \ \ *fwritef(c,F4,0,5/a)

\ \ \ \ *close(c/)

\ \ \ \ →1

\ \ \ \ *close(s/)

\ \ \ \ **

\begin{center}
\textcolor[rgb]{1,0,0}{Эхо-клиент}
\end{center}
main(/)

\ \ \ \ Os Oe Oa 

\ \ \ \ @+F1(16) @+F2(16) @+F3(5) @+F4(5)

\ \ \ \ @'192.168.203.1'>F2

\ \ \ \ @'echo\n'>F3

\ \ \ \ \hyperref[set_sockaddr_port]{*set\_sockaddr\_port}(F1,1195/)

\ \ \ \ \hyperref[set_sockaddr_ip]{*set\_sockaddr\_ip}(F1,F2/)

\ \ \ \ \hyperref[socket]{*socket}(1/s)

\ \ \ \ \hyperref[connect]{*connect}(s,F1/e)

\ \ \ \ *fwritef(s,F3,0,5/a)

\ \ \ \ *freadf(s,F4,0,5/a)

\ \ \ \ *fwritef(1,F4,0,5/a)

\ \ \ \ *close(s/)

\ \ \ \ **
\end{document}
